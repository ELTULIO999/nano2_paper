In June 2023 two senior students from the Electronics Engineering department at Universidad del Valle de Guatemala were granted scholarships 
to attend the ``2023 IEEE EDS Summer School R9" in Puebla, Mexico Throughout this enlightening summer program, a wide array of topics were 
addressed, ranging from insights of former students who work in the field, to several educational tools provided by one of the industry leaders; 
Synopsys. A significant message stood out: Nanochip design and fabrication industry is so complex, so dynamic, and so expensive due to infrastructure, 
human resources, etc. so that it becomes extremely challenging to almost impossible for academia to keep up with it.
During the program, the gap between industry capability and academic resources became evident as the alumni shared their experiences. 
Some of the former students highlighted that they had encountered setbacks in their projects due to challenges in installation or upkeep of some 
tools within the expansive realm of semiconductor technology. Notably, a few even mentioned that a lack of infrastructure and limited time had led to the 
unfortunate loss of grants they had secured for their projects.In the program, there were also discussions about new open-source tools such as Wokwi, 
designed to facilitate learning about logical circuitry. The Efabless initiative was another topic, that was introduce as a private semiconductor fabrication 
company, but the one that left the most significant impact was Tiny Tapeout because it closes de gap between the complexity behind the chip design and manufacturing 
in industry and the resources, time, and knowledge typically available in academia. This tool stood out for its adoption is further supported by a multi-purpose wafer 
and the cost-effectiveness of chip production. Typically, one of the challenges in academia involves acquiring the necessary computing power to run complex design tools. 
However, Tiny Tapeout has tackled this issue by providing its own cloud computing infrastructure, effectively reducing the demand for high computational resources. On 
the other hand, the installation of some of the highly advanced programs use in the industry are very complex using authentication servers to secure their tools, 
however Tiny Tapeout has manage to set their installation tool process very straightforward using only three steps. It’s worth mentioning that Tiny Tapeout has its 
own website where you can consult any doubt its installer might have, and it has several channels of communication. The aim of the Tiny Tapeout initiative is not 
to take over the semiconductor industry but to give students the knowledge and experience of working in this ample field, moreover, offer students the experience of taking a digital design all the way to semiconductor fabrication.
