This section introduces the Verilog implementation within the framework of Tiny Tapeout, a project 
dedicated to generating chip layouts from Verilog files. In this context, we will delineate the four 
distinct designs produced, providing comprehensive descriptions of the modules and their respective behaviors.

\subsubsection{Multy stage path for delay measurements}


The core of the module featured an approach to creating a ring oscillator. 
Although it was initially intended to utilize cascaded NOT gates, it was observed that the synthesizer, under the 
constraints of the Tiny Tapeout design flow, may replace these gates with buffers. 
As a result, the oscillatory behavior was expectedly altered. Nevertheless, the module's ring 
oscillator function was realized through a sequence of logical operations involving AND gates and 
inverters, forming a feedback loop. The logical signals EN and EN\_2 served as inputs to an AND\_2 
module, generating a waveform represented by W\_1. Subsequently, this waveform traversed a series of 
inverters (tt\_prim\_inv modules), resulting in the generation of W\_2, W\_3, and cyclically returning to W\_1. 
While this configuration may not conform precisely to the original design intent, it offers a valuable educational 
opportunity to explore gate delays and their implications in digital circuitry when compared to theoretical calculations.

\subsubsection{ASCII Text Printer Circuit}


A Verilog module was designed to function as a text printing system capable of displaying two distinct 
texts, which are selected through an external signal. The module employs an internal counter synchronized 
with the clock to determine the specific ASCII character displayed based on the selection signal and the 
counter value. The output is provided through the output pins in ASCII format.

\subsubsection{Implementation of the Pong game}

The Verilog code, pong\_neopixel.v, with the main module ``tt\_um\_pong\_neopixel", has been 
developed with the purpose of implementing a version of the Pong game on a Neopixel pixel 
matrix. This design has been conceived as an example of an interactive and playful application 
of programmable digital hardware. The "tt\_um\_pong\_neopixel" module consists of several 
inputs and outputs intended to control the game, including player input signals, start 
signals, and outputs to manage the Neopixel matrix, along with clock and reset signals. To 
ensure the stability of the input signals, "debounce" modules have been implemented. The game 
logic includes the management of the movement of the players' paddles and the ball, as well as 
collision detection and game reset when appropriate. In addition, logic has been developed to 
generate Neopixel signals that control the display on the matrix, allowing player interaction. 
This design also incorporates counters to track the sending of data to the matrix and 
appropriately selects whether LEDs should be turned on or off based on the position of the 
ball and paddles.

\subsubsection{Pulse Width Modulation Generator}


The Verilog code, pwm\_generator.v, with the main module "tt\_um\_pwm", is designed to 
generate a pulse width modulation (PWM) signal controlled by buttons. The module allows 
the duty cycle of the PWM signal to be increased or decreased via buttons. To ensure a 
reliable reading of the buttons, debounce logic is implemented that 
generates a slow clock signal (slow\_clk\_enable).