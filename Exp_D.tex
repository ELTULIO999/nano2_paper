The execution of the Tiny Tapeout chip design flow begins with a series of essential steps. First it is necessary to fork the example repository provided by the Tiny Tapeout project. Once this step is completed, enabling the repository's capability to perform actions is crucial, as it allows for the automation of subsequent processes. To ensure the project's proper execution, several key tasks must be carried out. First all Verilog files must be located in the "src" folder. Additionally, the "info.yaml" file must be filled out with relevant information, including the Verilog files that constitute the design, the main design module, the authors, a project description, and, if applicable, the clock frequency. These steps are fundamental to ensuring an effective design flow and the correct implementation of chips in Tiny Tapeout.