% 1st Paragraph, 
% ref [1] is:  “Semiconductor Leaders - IEEE IRDSTM.” 
The semiconductor industry -chip/microchip industry- holds a significant share in the market owing to its rapid strides in technological advancement. Over the years, there has been a noticeable trend of various companies pushing the limits of nanoscale technology, driving innovation within the sector [1]. This remarkable progress has been made possible primarily by two major aspects: on one hand there are  super specialised chip manufacturing hardware, that bring to life the digital representation of the chip. On the other hand there are very specialised software tools that are made by very specific software developing companies, who are responsible for making said digital representation of the chip. 

% 2st Paragraph, 
% ref [7] is: Week In Review: Semiconductor Manufacturing, Test
% ref [8] is: Intel is spending $20 billion to build two new chip plants in Arizona
Intel, Apple, AMD, Samsung and similar semiconductor industry giants have the financial resources to invest in these specialised tools and software, which are often costly due to their advanced capabilities and precision engineering [7-8]. These tools facilitate tasks such as layout design, simulation, verification, and manufacturing of semiconductor chips. By utilising such tools, these companies can explore innovative concepts, test them in virtual as well as physical environments, and fine-tune their designs before misproduction.

% 3st Paragraph, 
% ref [10] is: Design for Manufacturing Overview.
% ref [11] is: Synopsys, Cadence, Google And NVIDIA All Agree: Use AI To Help Design Chips,
% ref [12] is: What is Electronic Design Automation (EDA)? – How it Works | Synopsys.
Additionally, software development companies like Synopsys, Cadence, and Mentor Graphics play a crucial role in shaping the semiconductor landscape [11]. They continuously develop and refine software that cater to the intricate needs of chip designers and manufacturers [12]. These tools streamline workflows, improve design accuracy, and optimise manufacturing processes, ultimately leading to the production of high-quality chips that power a wide array of electronic devices, from smartphones, laptops, autonomous vehicles to sophisticated industrial equipment, all the way to life-critical biomedical equipment, satellites, and space shuttles [10]. 

% 4st Paragraph, 
% ref [13] is: Reshoring Semiconductor Manufacturing: Addressing the Workforce Challenge
% ref [15] is: A Vision and Strategy for the NSTC.
However, it's worth acknowledging that while the semiconductor industry thrives on innovation, learning about the intricacies of integrated circuit design and manufacturing processes can be very difficult [13,15]. This challenge is one of the key hurdles that stems from the gap in access to chip design and manufacturing processes between industry and academia. The tools and resources used by technology giants already mentioned are often proprietary and come with substantial financial barriers. This fination challenge creates a division between the cutting-edge practices employed in the industry and the educational resources available in academic settings [13].

% 5st Paragraph,
Students and researchers in academia might encounter difficulties in gaining hands-on experience with the most up-to-date tools and techniques used in the semiconductor industry. Moreover, access to advanced fabrication facilities and sophisticated design software can be limited, potentially hindering their ability to fully grasp the complexities of modern chip design. The rapid pace of innovation in the industry can also make it challenging for academic curricula to keep up with the latest developments, further emphasising the disparity between theoretical knowledge and practical application.

% 6st Paragraph,
% ref [16] is: Roadmapping of Nanoelectronics for the New Electronics Industry.
% ref [19] is: Tiny Tapeout
% ref [18] is: Energizing collaborative industry-academia learning: a present case and future visions
Efforts are being made to bridge this gap by fostering collaborations between industry and academia [16]. Some companies offer educational programs, internships, and partnerships with universities, enabling students and researchers to gain exposure to state-of-the-art tools and real-world design challenges [19]. Moreover, open-source initiatives and shared research facilities aim to provide broader access to resources that were once exclusive to industry giants. By narrowing the division between industry and academia, aspiring chip designers and researchers can better prepare themselves for the demands of the semiconductor landscape [18].

% 7st Paragraph,
As the semiconductor industry propels forward with innovation, the challenge lies in ensuring that the knowledge and expertise surrounding integrated circuit design and manufacturing are accessible to all. Bridging the gap between the tools and processes utilised in industry and those available in academic settings will be crucial for nurturing the next generation of skilled professionals who can contribute to the ongoing advancements in the semiconductor field. In this work, an open-source environment for a design flow for chip manufacturing will be deployed at  Universidad del Valle de Guatemala to shorten the gap between industry and academia. 